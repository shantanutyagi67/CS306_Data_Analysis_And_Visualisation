\documentclass[a4paper]{article}

\usepackage{fullpage} % Package to use full page
\usepackage{parskip} % Package to tweak paragraph skipping
\usepackage{tikz} % Package for drawing
\usepackage[fleqn]{amsmath}
\usepackage{hyperref}
\hypersetup{ 
    colorlinks=true,
    linkcolor=red,
    filecolor=magenta,      
    urlcolor=blue,
}
\urlstyle{same}

\usepackage{float}
\usepackage{multicol}
\usepackage{listings}
\lstset{language=Python}
\lstset{frame=lines}
\lstset{basicstyle=\footnotesize}

\usepackage{changepage}
\usepackage{algorithmicx}
\usepackage{algpseudocode}
\usepackage{subcaption}
\usepackage{mathtools}

\usepackage{biblatex}
\addbibresource{bibliography.bib}

\renewcommand\thesubsection{\thesection.\alph{subsection}}
\renewcommand\thesubsubsection{\thesubsection.\roman{subsubsection}}

\title{\textbf{CS 306\\ Assignment 1}}
\author{Name IDK\\IDK's ID}
\date{Due date: April 18, 2021}

\begin{document}

\maketitle

\section{Constraint optimization and modeling using linear algebra techniques}
The general equation of a circle is $(x - x_0)^2 + (y - y_0)^2 = r^2$, where $(x_0, y_0)$ is the circle centre and $r$ is the radius.
\par General equation of a conic is $ax^2 + bxy + cy^2 + dx + ey + f = 0$. A circle is a conic with certain constraints.

\subsection{Compute and state the values of $a, b, c, d, e, f$ in terms of the parameters: circle center, and the radius}
Expanding the equation of a circle, we get:
\begin{align*}
             & (x - x_0)^2 + (y - y_0)^2 = r^2                         \\
    \implies & x^2 - 2xx_0 + {x_0}^2 + y^2 - 2yy_0 + {y_0}^2 = r^2     \\
    \implies & x^2 - 2xx_0 + {x_0}^2 + y^2 - 2yy_0 + {y_0}^2 - r^2 = 0 \\
    \implies & x^2 + y^2 - 2x_0x - 2y_0y + {x_0}^2 + {y_0}^2 - r^2 = 0
\end{align*}
Comparing this with the general equation of a conic, we get:
\begin{equation*}
    \centering
    ax^2 + bxy + cy^2 + dx + ey + f = 0 \equiv x^2 + y^2 - 2x_0x - 2y_0y + {x_0}^2 + {y_0}^2 - r^2 = 0
\end{equation*}
\begin{align*}
    a & = 1    & b & = 0                      \\
    c & = 1    & d & =-2x_0                   \\
    e & =-2y_0 & f & ={x_0}^2 + {y_0}^2 - r^2
\end{align*}

\subsection{Outline the Direct Linear Transform (DLT) algorithm for computing least square estimate of $a, b, c, d, e, f$ for a given geometric 2-dimensional point data.}
% According to DLT, a matrix equation $\mathbf{X} =\mathbf {A} \mathbf {Y}$ can be solved for $\mathbf{A}$ using $\mathbf{A} =\mathbf{X} \mathbf{Y}^T (\mathbf{Y}\mathbf{Y}^T)^{-1}$
We will assume $\mathbf{X}$ to be the $x$ coordinates of the data, and $\mathbf{Y}$ to be the $y$ coordinates of the data. Thus in our case, we have:
$A = \begin{bmatrix}
        \vdots & \vdots & \vdots & \vdots & \vdots & \vdots \\
        X^2    & XY     & Y^2    & X      & Y      & 1      \\
        \vdots & \vdots & \vdots & \vdots & \vdots & \vdots
    \end{bmatrix}_{n\times 6}$,
$X = \begin{bmatrix}
        a \\b\\c\\d\\e\\f
    \end{bmatrix}_{6\times 1}$ and
$b = \begin{bmatrix}
        0 \\0\\0\\\vdots\\0
    \end{bmatrix}_{n\times 1}$.
\par For least squares solution, we find $\mathbf{X}$ that minimizes $\left\lVert \mathbf{A}\mathbf{X} - \mathbf{b}\right\rVert$.
\begin{align*}
                   & \mathbf{U} \mathbf{D}\mathbf{V}^T = SVD(\mathbf{A})                                     \\
                   & \left\lVert \mathbf{A}\mathbf{X}-\mathbf{b}\right\rVert = 0                             \\
    \implies       & \left\lVert \mathbf{U} \mathbf{D}\mathbf{V}^T\mathbf{X}-b \right\rVert=0                \\
    \implies       & \left\lVert \mathbf{D}\mathbf{V}^T \mathbf{X} - \mathbf{U}^T \mathbf{b}\right\rVert = 0 \\
    \text{Since, } & \mathbf{U}^T \mathbf{b} = 0                                                             \\
    \text{and }    & \mathbf{Y} = \mathbf{V}^T \mathbf{X}
\end{align*}
Thus the last column of $\mathbf{V}^T$ will contains values of $\mathbf{X}$ for minimum values of $\left\lVert \mathbf{A}\mathbf{X}-\mathbf{b}\right\rVert = 0$.

\subsection{We are given geometric point data to compute the parameters of a circle which will best fit the data. Construct and state the constraint matrix for this circle fitting problem.}
We have $ax^2 + bxy + cy^2 + dx + ey + f = 0$.
Let $\mathbf{X} = \begin{bmatrix}
        x \\y\\1
    \end{bmatrix}_{3 \times 1}$, and $\mathbf{C}= \begin{bmatrix}
        a & b/2 & d/2 \\ b/2&c&e/2\\ d/2&e/2&f
    \end{bmatrix}_{3 \times 3}$.
\par Therefore,
\begin{align*}
             & \mathbf{X}^T\mathbf{C}\mathbf{X}=0                                                                              \\
    \implies & \begin{bmatrix}
        x & y & 1
    \end{bmatrix} \begin{bmatrix}
        ax + by/2 + d/2 \\  bx/2 + cy + e/2 \\ dx/2  + ey/2 + f
    \end{bmatrix} = 0                                                       \\
    \implies & ax^2 + \frac{bxy}{2} + \frac{dx}{2} + \frac{bxy}{2} + cy^2 + \frac{ey}{x} + \frac{dx}{2} + \frac{ey}{2} + f = 0
\end{align*}
Using linear transform, $\mathbf{A}\mathbf{X}=0$. Thus,
\begin{align*}
                & (\mathbf{A}\mathbf{X})^T \mathbf{A}\mathbf{X} = 0 \\
    \implies    & \mathbf{X}^T(\mathbf{A}^T\mathbf{A})\mathbf{X}=0  \\
    \text{and } & \mathbf{X}^T\mathbf{C}\mathbf{X}=0                \\
\end{align*}
Here, $\mathbf{A}$ is  the design matrix, $\mathbf{A}^T\mathbf{A}$ is the scatter matrix, and $\mathbf{C}$ is the constraint matrix.
\par For a circle, $b^2-4ac<0$.
\par Thus, we need to find $\mathbf{X}$ such that
\begin{align*}
    \underset{\mathbf{X}}{\mathrm{argmin}} \begin{bmatrix}
        \mathbf{X}^T(\mathbf{A}^T\mathbf{A})\mathbf{X} \mid \mathbf{X}^T\mathbf{C}\mathbf{X} = -\phi
    \end{bmatrix}
\end{align*}
Using Lagrangian Multiplier,
\begin{align*}
             & \mathcal{L}(\mathbf{X})                                  = \mathbf{X}^T(\mathbf{A}^T\mathbf{A})\mathbf{X} - \lambda(\mathbf{X}^T\mathbf{C}\mathbf{X} + \phi) \\
    \implies & \mathcal{L}^\prime(\mathbf{X})                  = 0                                                                                                          \\
    \implies & \mathbf{X}^T(\mathbf{A}^T\mathbf{A})\mathbf{X}  = \lambda\mathbf{X}^T\mathbf{C}\mathbf{X} = \lambda\phi                                                      \\
    \implies & (\mathbf{A}^T\mathbf{A})\mathbf{X} = \lambda\mathbf{C}\mathbf{X}                                                                                             \\
    \implies & \frac{1}{\lambda}\mathbf{X} = (\mathbf{A}^T\mathbf{A})^{-1}\mathbf{C}\mathbf{X}
\end{align*}
Using EVD of $(\mathbf{A}^T\mathbf{A})^{-1}\mathbf{C}$, we find $\lambda^{-1}$, which is the required parameter.
\end{document}